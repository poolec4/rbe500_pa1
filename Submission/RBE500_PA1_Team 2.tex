\documentclass[10pt]{article}
\usepackage[letterpaper,margin=0.725in]{geometry}

\usepackage{fancyhdr}
\usepackage{amsmath}
\usepackage{mathtools}
\usepackage{hyperref}
\usepackage[english]{babel}
\usepackage{graphicx}
\usepackage{float}
\usepackage{caption}
\usepackage{amssymb}
\usepackage{caption}
\usepackage{amstext} % for \text macro
\usepackage{array}   % for \newcolumntype macro
\allowdisplaybreaks

\usepackage[T1]{fontenc}
\usepackage[numbered,framed]{matlab-prettifier}

\hypersetup{ colorlinks=true, linkcolor=blue}

\DeclareMathOperator{\atantwo}{atan2}
\newcolumntype{C}{>{$}c<{$}} % math-mode version of "l" column type

\pagestyle{fancy}
\lhead{RBE 500 - PA \#1 \newline Team 2: Peter Campellone, Aislin Hanscom, Christopher Poole}
\rhead{Due: 6/29/2021}

\begin{document}

\setlength{\abovedisplayskip}{6pt}
\setlength{\belowdisplayskip}{3pt}
\setlength{\abovedisplayshortskip}{4pt}
\setlength{\belowdisplayshortskip}{4pt}

\textbf{Package overview:}
\begin{itemize}
	\item Inside \texttt{catkin\_ws/src}, the main package is \texttt{scara\_robot}. It does not directly contain any nodes or launch files, but is a way to organize all of the other nodes.
	\begin{itemize}
		\item The \texttt{scara\_gazebo} package includes the launch files for the gazebo world.
		\item The \texttt{scara\_description} package includes the URDF files for the robot as well as the rviz launch files.
		\item The \texttt{gazebp\_publish} package includes the launch file to allow for the joint states to be published from gazebo.
		\item The \texttt{scara\_forward\_kinematics} folder is the pub/sub package that subscribes to the joint states, calculates the forward kinematics, and publishes the pose.
		\item The \texttt{scara\_inverse\_kinematics} folder is the service/client package that ingests a desired end effector pose and returns the joint position.
	\end{itemize}
\end{itemize}

\begin{enumerate}
	\item \textbf{Create the robot:}
	
	\begin{itemize}
		\item Spawn the robot to the ROS-Gazebo environment. Include and image of the robot.
		
		\item Include the robot definition file.
	\end{itemize}

	Using the URDF file format, we created a \texttt{scara.xacro} file which describes the SCARA robot that can be seen in the below figure. This robot is created using six links and five joints; see the tables below for the specific joints/links used and their functions. We additionally use a camera joint and link that is used to center the camera on our SCARA robot. The lengths of the links used in this SCARA robot are completely configurable but for this project arbitrary dimensions were used that allowed the user to observe the operation of the robot easily. Additionally the rotational limits of all of the joints (except for the fixed joints) are completely configurable as well (see the table below for current limits). We further added a \texttt{materials.xacro} URDF file to allow multiple colors to be implemented for all of the links so that it's easy to differentiate between them. 
	
	\begin{figure}[h]
	\begin{minipage}[h]{0.5\textwidth}
		\centering
		\includegraphics[width=0.8\textwidth]{figures/rviz_robot.png}
		\caption{SCARA robot in rviz}
	\end{minipage}
	\begin{minipage}[h]{0.5\textwidth}
		\centering
		\includegraphics[width=0.8\textwidth]{figures/gazebo_robot.png}
		\caption{SCARA robot in gazebo}
	\end{minipage}
	\end{figure}
	
	The following joint and link definitions were used:
	\begin{table}[h]
		\centering
		\begin{tabular}{|c|c|c|c|c|}
			\hline
			Joint Name & Joint Type & Joint Limits \\\hline
			Joint 1 & Continuous & $-180^\circ$ to $180^\circ$ \\
			Joint 2 & Fixed & N/A \\
			Joint 3 & Revolute & $-90^\circ$ to $90^\circ$ \\
			Joint 4 & Fixed & N/A \\
			Joint 5 & Prismatic & 0m to 1m \\\hline
		\end{tabular}
		\caption{\label{tab:widgets}Joints}
	\end{table}
	
	\begin{table}[H]
		\centering
		\begin{tabular}{|c|c|c|c|c|}
			\hline
			Link Name & Link Shape & Link Color & Link Purpose \\\hline
			Link 1 & Cylinder & Orange & Connects to base frame \\
			Link 2 & Cylinder & Red & Models $\theta_1$ Rotation \\
			Link 3 & Box & Orange & Models $\theta_1$ Rotation with link 2\\
			Link 4 & Cylinder & Blue & Models $\theta_2$ Rotation \\
			Link 5 & Box & Red & Models $\theta_2$ Rotation with link 4\\
			Link 6 & Box & Green & Translates along $d_3$ \\\hline
		\end{tabular}
		\caption{\label{tab:widgets}Links}
	\end{table}

	To begin the simulation:
	\begin{itemize}
		\item \texttt{roslaunch scara\_gazebo scara\_world.launch}
	\end{itemize}	
	
	Once the simulation begins, a force/torque can be applied to the joints to induce rotation/translation, as seen below (link 3 would require a constant force to keep from dropping down):
	
	\begin{figure}[h]
		\centering
		\includegraphics[width=0.5\textwidth]{figures/gazebo_robot_moved.png}
		\caption{SCARA robot in gazebo after forces/torques applied}
	\end{figure}

	The robot description file and gazebo/rviz launch files can be found in the \texttt{scara\_description} and \texttt{scara\_gazebo} folders. 

	\item \textbf{Forward Kinematics:}
	Implement a FK node that
	
	\begin{itemize}	
		\item Subscribes to the joint values topic and reads them from the Gazebo simulator.
		\item Calculates the pose of the end-effector.
		\item Publishes the pose as a ROS topic (inside the callback function that reads the joint values).
	\end{itemize}
	Print the resulting pose to the terminal using the \texttt{rostopic echo} command and include a screenshot of the results.
	\\
	
	We begin by assigning coordinate frames to the manipulator:
	
	\begin{figure}[h]
		\centering
		\includegraphics[width=0.6\textwidth]{figures/rrp_manipulator_reference_frames.png}
	\end{figure}
	
	We can then formulate the DH parameters coordinating to the links:
	
	\begin{center}
	\begin{tabular}{ c | C | C | C | C }
		Link & \theta_i & d_i & a_i & \alpha_i \\
		\hline 
		1 & \theta_1^* & d_1 & l_1 & 0 \\
		2 & \theta_2^* & 0 & l_2 & 0 \\
		3 & 0 & -d_3^* & 0 & 0 \\
	\end{tabular}
	\end{center}
	
	Next, we can calculate the transformations for each frame:
	
	\begin{align*}
		T_{i+1}^i &= \text{Rot}_z(\theta_i) \text{Trans}_z(d_i) \text{Trans}_x(a_i) \text{Rot}_x(\alpha_i) = \begin{bmatrix}
		\cos\theta_i & -\sin\theta_i \cos\alpha_i &\sin\theta_i \sin\alpha_i & a_i \cos\theta_i \\
		\sin\theta_i & \cos\theta_i \cos\alpha_i & -\cos\theta_i \sin\alpha_i & a_i \sin\theta_i \\
		0 & \sin\alpha_i & \cos\alpha_i & d_i \\
		0 & 0 & 0 & 1
		\end{bmatrix}
		\\
		T_{2}^1 &= \begin{bmatrix}
		1 & 0 & 0 & 0 \\
		0 & 1 & 0 & 0 \\
		0 & 0 & 1 & 0 \\
		0 & 0 & 0 & 1
		\end{bmatrix}
		\\
		T_{2}^1 &= \begin{bmatrix}
		\cos\theta_1 & -\sin\theta_1 \cos 0 &\sin\theta_1 \sin 0 & l_1 \cos\theta_1 \\
		\sin\theta_1 & \cos\theta_1 \cos 0 & -\cos\theta_1 \sin 0 & l_1 \sin\theta_1 \\
		0 & \sin 0 & \cos 0 & d_1 \\
		0 & 0 & 0 & 1
		\end{bmatrix} = \begin{bmatrix}
		\cos\theta_1 & -\sin\theta_1 & 0 & l_1 \cos\theta_1 \\
		\sin\theta_1 & \cos\theta_1 & 0 & l_1 \sin\theta_1 \\
		0 & 0 & 1 & d_1 \\
		0 & 0 & 0 & 1
		\end{bmatrix}
		\\
		T_{3}^2 &= \begin{bmatrix}
		\cos\theta_2 & -\sin\theta_2 \cos 0 &\sin\theta_2 \sin 0 & l_2 \cos\theta_2 \\
		\sin\theta_2 & \cos\theta_2 \cos 0  & -\cos\theta_2 \sin 0 & l_2 \sin\theta_2 \\
		0 & \sin 0 & \cos 0 & 0 \\
		0 & 0 & 0 & 1
		\end{bmatrix} = \begin{bmatrix}
		\cos\theta_2 & -\sin\theta_2 & 0 & l_2 \cos\theta_2 \\
		\sin\theta_2 & \cos\theta_2 & 0 & l_2 \sin\theta_2 \\
		0 & 0 & 1 & 0 \\
		0 & 0 & 0 & 1
		\end{bmatrix}
		\\
		T_{4}^3 &= \begin{bmatrix}
		\cos 0 & -\sin 0 \cos 0 &\sin 0 \sin 0 & 0 \cos 0 \\
		\sin 0 & \cos 0 \cos 0 & -\cos 0 \sin 0 & 0 \sin 0 \\
		0 & \sin 0 & \cos 0 & -d_3 \\
		0 & 0 & 0 & 1
		\end{bmatrix} = \begin{bmatrix}
		1 & 0 & 0 & 0 \\
		0 & 1 & 0 & 0 \\
		0 & 0 & 1 & -d_3 \\
		0 & 0 & 0 & 1
		\end{bmatrix}
	\end{align*}
	
	The combined transformation of the end effector is:
	\begin{align*}
		T_4^0 = T_1^0 T_2^1 T_3^2 T_4^3
	\end{align*}
	
	This calculation is used in the forward kinematic function \texttt{calc\_homogeneous\_transform(q)}:
	
\begin{lstlisting}[style=Matlab-editor,basicstyle=\mlttfamily,escapechar=`]
def calc_homogeneous_transform(q): # calculate the homogeneous transform from the base frame to EE
	# q = [th1, th2, d3]
	th1 = q[0]
	th2 = q[1]
	d3 = q[2]
	
	T1_0 = np.matrix([[1,0,0,0],[0,1,0,0],[0,0,1,0],[0,0,0,1]]) # frame 1 w.r.t 0
	T2_1 = np.matrix([[math.cos(th1),-math.sin(th1),0,l1*math.cos(th1)],[math.sin(th1),math.cos(th1),0,l1*math.sin(th1)],[0,0,1,d1],[0,0,0,1]])	# frame 2 w.r.t 1 
	T3_2 = np.matrix([[math.cos(th2),-math.sin(th2),0,l2*math.cos(th2)],[math.sin(th2),math.cos(th2),0,l2*math.sin(th2)],[0,0,1,0],[0,0,0,1]])	# frame 3 w.r.t 2 
	T4_3 = np.matrix([[1,0,0,0],[0,1,0,0],[0,0,1,-d3],[0,0,0,1]]) # frame 4 w.r.t 3
	
	T_EE = T1_0.dot(T2_1).dot(T3_2).dot(T4_3)
	
	return T_EE
\end{lstlisting}
	

	To run and test the subscriber/publisher function:
	
	\begin{itemize}
		\item Run \texttt{roslaunch scara\_gazebo scara\_world.launch}
		\item In a new window \texttt{roslaunch gazebo\_publish gazebo\_publish.launch} which allows the joint states to be published.
		\item In a new window \texttt{rostopic list} should show /scara/joint\_states
		\item \texttt{rostopic echo /scara/joint\_states} should show the joint states printing
		\item \texttt{rosrun scara\_forward\_kinematics configuration\_to\_operational\_sub.py} will run the program that subscribes to the joint states, calculates the forward kinematics, and publishes the end-effector pose back to the \texttt{Pose} topic. The print out can be seen below:
		
		\begin{figure}[h]
			\centering
			\includegraphics[width=0.7\textwidth]{figures/sub_pub_print_results.png}
		\end{figure}
	
		\item We can also see the \texttt{Pose} topic getting published by running \texttt{rostopic echo /scara/pose}
		
		\begin{figure}[h]
			\centering
			\includegraphics[width=0.18\textwidth]{figures/pose_echo.png}
		\end{figure}
		
		Note: the above shows the robot in the home position (i.e on the $x-z$ plane)
	\end{itemize}

	\item \textbf{Inverse Kinematics:}
	
	\begin{itemize}
		\item Implement an IK node (separate node) that has a service client that take a desired pose of the end effector and returns joint positions as a response.
	
		\item Test your node with \texttt{rosservice call}. Include a screenshot of the results.
	\end{itemize}

	The following definitions can be used to calculate the inverse kinematics:
	\begin{figure}[H]
		\centering
		\includegraphics[width=0.75\textwidth]{figures/rrp_IK.png}
	\end{figure}
	
	The large right triangle can be used to calculate the following:
	\begin{align*}
		r &= \sqrt{p_x^2 + p_y^2} \\
		A &= \frac{p_x}{r} = \cos\alpha \Rightarrow
		\sin \alpha = \pm \sqrt{1 - A^2} \Rightarrow \alpha = \atantwo\left(\pm \sqrt{1-A}, A\right)
	\end{align*}
	\\
	The law of cosines can be used on the other triangle to calculate $\beta$ and $\theta_1$:
	\begin{align*}
		l_2^2 &= r^2 + l_1^2 - 2 r l_1 \cos\beta \\
		\cos\beta & = \frac{r^2 + l_1^2 - l_2^2}{2 r l_1} = C \Rightarrow \sin\beta = \pm\sqrt{1-C^2} \\
		\beta &= \atantwo\left(\pm\sqrt{1-C^2}, C\right) \\
		\theta_1 &= \alpha - \beta
	\end{align*}
	\\
	The law of cosines can be used again to calculate $\theta_1$:
	\begin{align*}
		r^2 &= l_1^2 + l_2^2 - 2 l_1 l_2 \cos(180-\theta_2) \\
		r^2 &= l_1^2 + l_2^2 + 2 l_1 l_2 \cos(\theta_2) \\
		\cos\theta_2 & = \frac{r^2 - l_1^2 - l_2^2}{2 l_1 l_2} = D \Rightarrow \sin\theta_2 = \pm\sqrt{1-D^2} \\
		\theta_2 &= \atantwo\left(\pm\sqrt{1-D^2}, D\right)
	\end{align*}
	\\
	Lastly, $d3$ is calculated simply as follows:
	\begin{align*}
		d_3 = d_1 - p_z
	\end{align*}
	\\
	The above equations are included in the server \texttt{inverse\_server.cpp}:
\begin{lstlisting}[style=Matlab-editor,basicstyle=\mlttfamily,escapechar=`]
double D = - (((l1*l1)+(l2*l2)-(x*x+y*y))/(2*l1*l2));
double C = (((l1*l1)+x*x+y*y-(l2*l2))/(2*l1*sqrt(x*x+y*y)));

double B = sqrt(1-D*D);
double E = sqrt(1-C*C);

double F = x/sqrt(x*x+y*y);
double G = sqrt(1-F*F);	
double alpha = atan2(G, F);

res.theta1 = alpha-atan2(E, C);
res.theta2 = atan2(B, D);
res.d3 = d1 - z;
\end{lstlisting}

To test the node:
\begin{itemize}
	\item \texttt{rosrun scara\_inverse\_kinematics inverse\_server}
	\item \texttt{rosrun scara\_inverse\_kinematics inverse\_client X Y Z}
	
	\begin{figure}[H]
		\centering
		\includegraphics[width=0.75\textwidth]{figures/client_print.png}
	\end{figure}

	\item The server will then print out the incoming message and send back the response. The following shows the home position (i.e. all joints at 0):
	
	\begin{figure}[H]
		\centering
		\includegraphics[width=0.75\textwidth]{figures/server_print.png}
	\end{figure}
	
	\item the server can also be tested using \texttt{rosservice}:
	
	\begin{figure}[H]
		\centering
		\includegraphics[width=0.75\textwidth]{figures/rosservice_call.png}
	\end{figure}
	
	which results in the following:
	
	\begin{figure}[H]
		\centering
		\includegraphics[width=0.75\textwidth]{figures/server_rosservice.png}
	\end{figure}
\end{itemize}

\end{enumerate}
\end{document}



